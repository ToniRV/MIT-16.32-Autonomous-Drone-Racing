%!TEX root = ../main.tex
\begin{abstract}
    Classical implementations of Visual-Inertial Odometry (VIO) algorithms ignore semantic information of the scene, as they rely solely on sparse landmarks.
    Nevertheless, recent work has shown the advantage of using richer representations of the scene, such as 3D meshes, to extract higher-level information such as structural regularities.
    In this work, we show that a 3D mesh of the scene can be further utilized to accommodate semantic information, which enhances the mapping side of a classical VIO beyond a sparse and uninformative point-cloud.
    Towards this end, we use recent work on semidefinite programming and conditional random fields to generate semantic information in real-time on a single-core CPU.
\end{abstract}

% Keywords appear just beneath the abstract. Use only for final version.
\begin{IEEEkeywords}
Vision-Based Navigation, Semantic Segmentation.
\end{IEEEkeywords}

\vspace{-2em}
\section*{Supplementary Material}
Videos of the experiments: \TODO{Add video url}


