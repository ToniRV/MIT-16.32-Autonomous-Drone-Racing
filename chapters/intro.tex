%!TEX root = ../main.tex

\section{Introduction}
\label{sec:introduction}

% Drop letter for first word of the Introduction
% Use only for final version.
 \IEEEPARstart{T}{he} advent of drone racing competitions, both manned and unmanned, requires the estimation of the fastest possible trajectory through a given race track.
 Finding this optimal trajectory can be useful for human pilots, and their tacticians, to evaluate their performance and improve upon it.
 For example, the teams in the Red Bull Air Race \cite{redbull} make use of such information to train and perfect their maneuvers.
 In this same way, drones can autonomously follow a dynamically feasible trajectory with high accuracy.
 This is especially interesting for autonomous drone races such as the Alpha Pilot race \cite{AlphaPilot}.
 Therefore, being able to infer the optimal feasible trajectory could potentially make fully-autonomous drones beat human pilots.
 We are still not there, but great progress in this direction is being made \cite{delmerico2019we}.

{\bf Contributions.}
\begin{itemize}
  \item Formulation of the minimum-time problem for the specific problem of drone-racing through multiple gates.
  \item Presentation of a means to solve the problem using readily available software.
\end{itemize}

{\bf Paper Structure.}
%\Cref{sec:related_work} introduces related work with a similar objective as ours.
Section~\ref{sec:mathematical_formulation} presents the mathematical model of our quadcopter, and discusses the assumptions made.
Section~\ref{ssec:optimization_problem} shows the optimal problem formulation.
Section~\ref{sec:results} reports and discusses the experimental results.
