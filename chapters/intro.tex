%!TEX root = ../main.tex

\section{Introduction}
\label{sec:introduction}

% Drop letter for first word of the Introduction
% Use only for final version.
 \IEEEPARstart{T}{he} advent of drone racing competitions such as \TODO{cite alpha pilot} requires the estimation of the fastest possible trajectory through a given race track.
 Finding this optimal trajectory can be useful for human pilots, and their tacticians, to evaluate their performance and improve upon it.
 For example, the teams in the Red Bull Air Race \TODO{cite redbull and SH} make extensive use of such information to train and perfect their manouevers.
 In this same way, drones can autonomously follow a dynamically feasible trajectory with high accuracy.
 Therefore being able to infer the optimal feasible trajectory could potentially make fully-autonomous drones beat human pilots.
 We are still not there, but great progress in this direction is being made.

{\bf Contributions.}
\begin{itemize}
  \item Formulation of the minimum-time problem for the specific problem of drone-racing through multiple gates.
  \item Presentation of a means to solve the problem using readily available software.
\end{itemize}

{\bf Paper Structure.}
Section~\ref{sec:mathematical_formulation} presents the mathematical formulation of our approach, and discusses the implementation of
Section~\ref{sec:results} reports and discusses the experimental results and comparison against related work. Section~\ref{sec:conclusions} concludes the paper.
