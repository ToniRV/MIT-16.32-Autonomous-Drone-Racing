%!TEX root = ../main.tex

\section{Conclusion}
\label{sec:conclusions}

In this paper, we have shown a VIO algorithm capable of building a 3D mesh of the scene without relying on extra regularization steps, but instead enforcing structural constraints extracted from the mesh.
We have also presented a way to incrementally build a 3D mesh while restricting it to the time-horizon of the optimization problem.
Therefore, the mesh spans multiple viewpoints, thereby covering an extended area; yet the size of the mesh remains bounded, allowing for real-time operation.

After evaluation of our approach, we have seen that enforcing co-planarity constraints between landmarks provides more accurate state \TODO{and mesh estimates} than simply ignoring these structural regularities.
In particular, the state estimation improves its global consistency by 26\% (Absolute Translation Error), while its local consistency improves by up to 50\% (Relative Position Error in translation), in scenes with structural regularities.
 Moreover, our proposed VIO algorithm surpasses in localization accuracy the state-of-the-art in scenes exhibiting structural regularities.
We also show that structural constraints improve the accuracy of the mesh by up to 7\%.

Finally, while the results are promising, we are not yet enforcing higher level regularities (such as parallelism or orthogonality) between planes.
Therefore, these improvements could be even larger, potentially rivaling pipelines enforcing loop-closures.
%Moreover, in the limit, the 3D mesh would achieve a quality similar to the meshes used in computer graphics.
%To achieve this, an approach could be to incrementally build the 3D mesh such that its manifold property is preserved, for example by using the work of \cite{} \TODO{cite}.

%We have hence confirmed, in real experiments, the conclusions drawn by previous authors \cite{Henein2017iros} about the benefits of using structural regularities.
%We have also implemented, to the best of our knowledge, the first VIO pipeline to use Structureless \cite{Carlone14icra} and Projection factors (both monocular and stereo) simultaneously.
%And hence the first to also show the benefits of using Structureless, Projection and Regularity factors in the same optimization problem.

