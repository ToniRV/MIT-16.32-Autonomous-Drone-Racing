%!TEX root = ../main.tex

\section{Approach}
\label{sec:mathematical_formulation}

\TODO{Apprach}
We consider a stereo visual-inertial system and adopt a \emph{keyframe}-based approach~\cite{Leutenegger15ijrr}.
This section describes our VIO front-end and back-end.
Our front-end proceeds by building a 2D Delaunay triangulation over the 2D keypoints at each keyframe.
Then, the VIO back-end estimates the 3D position of each 2D keypoint, which we use to project the 2D triangulation into a 3D mesh.
While we incrementally build the 3D mesh, we restrict the mesh to the time-horizon of the VIO optimization, which we formulate in a fixed-lag smoothing framework~\cite{Qin17arxiv, Carlone17icra-vioAttention}.
The 3D mesh is further used to extract structural regularities in the scene that are then encoded as constraints in the optimization problem.

%%%%%%%%%%%%%%%%%%%%%%%%%%%%%%%%%%%%%%%%%%%%%%%%%%%%%%%%%%%%%%%%%%%%%%%%%%%%%%%%%%%%
\subsection{Front-end}
\label{ssec:frontend}

Our front-end has the same components as a keyframe-based indirect visual-inertial odometry pipeline\cite{Leutenegger15ijrr, Blosch15iros}, but it also incorporates a module to generate a 3D mesh, and a module to detect structural regularities from the 3D mesh.
We refer the reader to \cite[Sec. 4.2.1]{RosinolMT} for details on the standard modules used, and we focus here instead on the 3D mesh generation and regularity detection.

\subsubsection{3D Mesh Generation}
\label{sssec:3d_mesh_generation}

Building a consistent 3D Mesh of the environment using a sparse point cloud from VIO is difficult because (i) the 3D positions of the landmarks are noisy, and some are outliers; (ii) the density of the point cloud is highly irregular; (iii) the point cloud is constantly morphing: points are being removed (marginalized) and added, while the landmarks' positions are being updated at each optimization step.
Therefore, we avoid performing a 3D tetrahedralisation directly from the sparse 3D landmarks, which would require expensive algorithms, such as space carving \cite{Pan2009proforma}.

\subsubsection{3D Mesh Propagation}
While some algorithms update the mesh for a single frame \cite{Greene17iccv,Teixeira16iros}, we attempt to maintain a mesh over the receding horizon of the fixed-lag smoothing optimization problem (\cref{ssec:backend}), which contains multiple frames.
The motivation is three-fold: (i) A mesh spanning multiple frames covers a larger area of the scene, which provides more information than just the immediate field of view of the camera. (ii) We want to capture the structural regularities affecting any landmark in the optimization problem.
(iii) Anchoring the 3D mesh to the time-horizon of the optimization problem also bounds the memory usage, as well as the computational complexity of updating the mesh.

\paragraph{Temporal propagation} deals with the problem of updating the 3D mesh when new keypoints appear and/or old ones disappear in the new frame.

\paragraph{Spatial propagation} deals with the problem of updating the global 3D mesh when a new local 3D mesh is available, and when old landmarks are marginalized from the optimization's time-horizon.
We solve the first problem by merging the new local 3D mesh to the previous (global) mesh, by ensuring no duplicated 3D faces are present.

\subsubsection{Regularity Detection}
\label{sssec:regularity_detection}

\subsubsection{Data Association}
\label{sssec:data_association}

\subsection{Back-end}
\label{ssec:backend}

\subsubsection{State Space}
\label{sssec:state_space}

If we denote the set of all keyframes up to time $t$ by $\calK_t$, the state of the system $\mathbf{x}_i$ at keyframe $i\in\calK_t$, is described by the IMU orientation $\R_i$,
 position $\tran_i$, velocity $\vel_i$ and biases $\bias_i$:
\beq
\mathbf{x}_i \doteq [\R_i,\tran_i,\vel_i,\bias_i],
\eeq
where the pose $(\R_i, \tran_i) \in \SEthree$, $\vel_i \in \Real^3$, and $\bias_i = [\bias^g_i \;\; \bias^a_i] \in \Real^6$, and $\bias^g_i, \bias^a_i \in \Real^3$ are the gyroscope and accelerometer biases, respectively.

We will only estimate the 3D positions $\landmark_l$ for a subset $\Lambda_t$ of all landmarks $\LandmarkSet_t$ visible up to time $t$: $\{\landmark_l\}_{l\in\Lambda_t}$, where $\Lambda_t \subseteq \LandmarkSet_t$.
We will avoid optimizing over the rest of the landmarks $\LandmarkSet_t\setminus\Lambda_t$ by using a structureless approach, as defined in \cite[Sec. VII]{Forster17troOnmanifold}, which circumvents the need to add the landmarks' positions as variables in the state.
This allows to trade accuracy for speed; as the optimization’s complexity increases with the number of variables to be estimated.

The set $\Lambda_t$ corresponds to the landmarks which we consider to satisfy a structural regularity.
In particular, we are interested in co-planarity regularities, which we introduce in \cref{sssec:regularity_constraints}.
Since we need the explicit landmark variables to formulate constraints on them, we avoid using a structureless approach for these landmarks.

Finally, the co-planarity constraints between the landmarks $\Lambda_t$ require the modelling of the planes $\PlaneSet_t$ in the scene.
Therefore, the variables to be estimated comprise the state of the system $\{\mathbf{x}_i\}_{i\in\calK_t}$, the landmarks which we consider to satisfy structural regularities $\{\landmark_l\}_{l\in\Lambda_t}$, and the planes $\{\Plane_\pi\}_{\pi\in\PlaneSet_t}$.
The variables to be estimated at time $t$ are:

\begin{equation}
  \calX_t \doteq \displaystyle\left\{\mathbf{x}_i, \landmark_l, \Plane_\pi\right\}_{i\in\calK_t, l\in\Lambda_t, \pi\in\Pi_t}
  \label{eq:state_vector}
\end{equation}

Since we are taking a fixed-lag smoothing approach for the optimization, we limit the estimation problem to the sets of variables that depend on the keyframes in a time-horizon of size $\TimeWindow$.
To avoid cluttering the notation, we skip the dependence of the sets $\calK_t$, $\Lambda_t$ and $\Pi_t$ on the parameter $\TimeWindow$.

By reducing the number of state variables to a given window of time $\TimeWindow$, we will make the optimization problem tractable and solvable in real-time.

\subsubsection{Measurements}
\label{sssec:measurements}

The input for our system consists on measurements from the camera and the IMU.
We define the image measurements at keyframe $i$ as $\meascam_i$ .
The camera can observe multiple landmarks $l$, hence $\meascam_i$ contains multiple image measurements $\mathbf{z}_{i}^{l}$,
 where we distinguish the landmarks that we will use for further structural regularities $\mathbf{z}_{i}^{l_{c}}$ (where the index $c$ in $l_c$ stands for constrained landmark),
  and the landmarks that will remain as structureless $\mathbf{z}_{i}^{l_s}$ (where the $s$ in the index of $l_s$ stands for structureless).
We represent the set of IMU measurements acquired between two consecutive keyframes $i$ and $j$ as  $\measimu_\subimu$.
Therefore, we define the set of measurements collected up to time $t$ by $\mathcal{Z}_t$:
\begin{equation}
  \mathcal{Z}_t \doteq \{\meascam_i, \measimu_\subimu\}_{\indmeas \in \mathcal{K}_t}.
  \label{eq:measurements}
\end{equation}


\subsubsection{Factor Graph Formulation}
\label{sssec:factor_graph}

We want to estimate the posterior probability $p(\calX_t|\mathcal{Z}_t)$ of our state $\calX_t$, as defined in \cref{eq:state_vector}, using the set of measurements $\mathcal{Z}_t$, defined in \cref{eq:measurements}.
Using standard independence assumptions between measurements and states, we arrive to the formulation in \cref{eq:factor_form}, where we grouped the different terms in factors $\phi$:
\begin{subequations}\label{eq:factor_form}
  \begin{align}
    &p(\calX_t|\mathcal{Z}_t) \overset{(a)}{\propto} p(\mathcal{Z}_t|\calX_t)p(\calX_t) \nonumber\\
    &= \phi_0(\mathbf{x}_0)\prod_{l_c\in\Lambda_t}\prod_{\pi\in\Pi_t}\!\phi_{\mathcal{R}}(\bm{\rho}_{l_c}, \bm\pi_\pi)^{\delta(l_c, \pi)} \label{eq:factor_form_a}\\
    &\quad\quad\quad\quad\prod_{(i,j)\in\calK_t}\!\!\!\phi_{\text{IMU}}(\mathbf{x}_i, \mathbf{x}_j) \label{eq:factor_form_b}\\
    &\prod_{i\in\calK_t}\prod_{l_c\in\meascam_i^c}\!\phi_{l_c}(\mathbf{x}_i, \bm{\rho}_{l_c}) \!\!\prod_{i\in\calK_t}\prod_{l_s\in\meascam_i^s}\!\phi_{l_s}(\mathbf{x}_i) \label{eq:factor_form_c}
  \end{align}
\end{subequations}
where we apply the Bayes rule in (a), and ignore the normalization factor over the measurements since it will not influence the result (\cref{sssec:map_estimation}).

\Cref{eq:factor_form_a} corresponds to the prior information we have over the state $p(\calX_t)$.
In this term, we encode regularity constraints between landmarks $\bm\rho_{l_c}$ and planes $\pi$, which we denote by $\phi_{\mathcal{R}}$.
We also introduce the data association term $\delta(l_c, \pi)$, which returns a value of 1 if the landmark $l_c$ is associated to the plane $\pi$, 0 otherwise.
We explain in \cref{sssec:data_association} how the data association is done.
The factor $\phi_0$ represents a prior on the first pose of the optimization's time-horizon.

In \cref{eq:factor_form_b}, we have the factor corresponding to the IMU measurements, which depends only on the consecutive keyframes $(i, j)\in\calK_t$.

Finally, \cref{eq:factor_form_c} encodes the factors corresponding to the camera measurements.
Since we want to distinguish between landmarks that are involved in structural regularities ($l_c$) and landmarks that are not ($l_s$), we split the product over $C_i$;
where we write $l_s\in \meascam_i^s$ or $l_c\in \meascam_i^c$ when a landmark $l_s$ or $l_c$, respectively, is seen at keyframe $i$ by the camera. Note that $\meascam_i = \meascam_i^c \cup \meascam_i^s$ and $\meascam_i^c \cap \meascam_i^s = \emptyset$.

In \cref{fig:factor_graph_s_p_r_1}, we use the expresiveness of factor graphs \cite{Kschischang01it} to show the actual dependencies between the variables in \cref{eq:factor_form}\footnote{We will use the notation proposed in \cite{Dietz10directed} to represent the factor graph.}.

\begin{figure}[h]
  \centering
  \begin{tikzpicture}
    % X nodes
    \node[latent] (x0) {$\mathbf{x}_0$};
    \node[latent, right= of x0, xshift=\state] (x1) {$\mathbf{x}_1$};
    \node[latent, right= of x1, xshift=\state] (x2) {$\mathbf{x}_2$};

    % L nodes
    \node[latent, below=0.5 of x0, xshift=1.414\state] (l0) {$\bm\rho_{l_0}$};
    \node[latent, below=0.5 of x1, xshift=1.414\state] (l1) {$\bm\rho_{l_1}$};
    \node[latent, below=0.5 of x2, xshift=1.414\state] (l2) {$\bm\rho_{l_2}$};

    % Pi nodes
    \node[latent, below=0.25 of l0, xshift=1.414\state] (pi0) {$\bm\pi_0$};

    % Prior on X0
    \factor[above=of x0, xshift=0\state] {x0prior} {above:$\phi_{0}$} {x0} {}; %

    % Pre-integrated IMU
    % For X0 - X1
    \factor[right=of x0, xshift=0.5\state] {x0-x1} {above:$\phi_{IMU}$} {x0, x1} {}; %
    % For X1 - X2
    \factor[right=of x1, xshift=0.5\state] {x1-x2} {above:$\phi_{IMU}$} {x1, x2} {}; %

    % Landmark observations
    % For L0
    \factor[below=0.18 of x0, xshift=0.707\state] {x0-l0} {left:$\phi_{l_c}$} {x0, l0} {}; %
    \factor[below=0.18 of x1, xshift=-0.650\state] {x1-l0} {right:$\phi_{l_c}$} {x1, l0} {}; %

    % For L1
    \factor[below=0.18 of x1, xshift=0.707\state] {x1-l1} {right:$\phi_{l_c}$} {x1, l1} {}; %

    % For L2
    \factor[below=0.18 of x2, xshift=0.707\state] {x2-l2} {right:$\phi_{l_c}$} {x2, l2} {}; %
    \factor[below=0.18 of x2, xshift=-0.707\state] {x1-l2} {right:$\phi_{l_c}$} {x1, l2} {}; %

    % For structureless
    \factor[above=0.8of x1, xshift=-0.707\state] {x0-x1-x2} {left:$\phi_{l_s}$} {x0, x1,x2} {}; %
    \factor[above=0.8of x2, xshift=-0.707\state] {x1-x2} {left:$\phi_{l_s}$} {x1,x2} {}; %

    % For plane
    % plane-landmarks
    \factor[below=0.1 of l0, xshift=0.807\state] {pi0-l0-l1} {left:$\phi_{\mathcal{R}}$} {l0, pi0} {}; %
    \factor[below=0.1of l1, xshift=-0.707\state] {pi0-l1} {above:$\phi_{\mathcal{R}}$} {l1, pi0} {}; %
    \factor[below=0.1of l2, xshift=-2.3\state] {pi0-l2} {right:$\phi_{\mathcal{R}}$} {l2, pi0} {}; %
  \end{tikzpicture}
  \caption{VIO factor graph combining Structureless ($\phi_{l_s}$), Projection ($\phi_{l_c}$) and Regularity ($\phi_{\mathcal{R}}$) factors (SPR).
The factor $\phi_{\mathcal{R}}$ encodes relative constraints between a landmark $l_i$ and a plane $\pi_0$.}
  \label{fig:factor_graph_s_p_r_1}
\end{figure}

\subsubsection{MAP Estimation}
\label{sssec:map_estimation}
Since we are only interested in the most likely state $\calX_t$ given the measurements $\mathcal{Z}_t$, we calculate the \emph{maximum a posteriori} (MAP) estimator $\calX_t^{\text{MAP}}$.
Maximizing $\calX_t^{\text{MAP}}$ is nevertheless not as convenient as minimizing the negative logarithm of the posterior probability, which, using \cref{eq:factor_form}, simplifies to \cref{eq:simplified_map} for zero-mean Gaussian noise:

\begin{equation}
  \begin{split}
   &\calX_t^{\text{MAP}} = \arg\min_{\calX_t} \normsq{\mathbf{r}_0}{\Sigma_0} + \!\!\!\sum_{l_c\in\Lambda_t}\!\sum_{\pi\in\Pi_t}\delta(l_c, \pi)\normsq{\mathbf{r}_{\mathcal{R}}}{\Sigma_{\mathcal{R}}} \\
                  &+ \!\!\!\!\!\sum_{(i,j)\in\calK_t}\!\!\!\normsq{\mathbf{r}_{\measimu_\subimu}}{\Sigma_{ij}} \!\!\! + \!\!\! \sum_{i\in\calK_t}\!\!\Bigg\{\!\sum_{l_c\in\meascam_i} \normsq{\mathbf{r}_{\mathcal{C}_{i}^{l_c}}}{\Sigma_\mathcal{C}} \!\!\! + \!\!\! \sum_{l_s\in\meascam_i} \normsq{\mathbf{r}_{\mathcal{C}_{i}^{l_s}}}{\Sigma_\mathcal{C}}\!\!\Bigg\} \\
  \end{split}
  \label{eq:simplified_map}
\end{equation}
where $\mathbf{r}$ represents the residual errors, and $\mathbf{\Sigma}$ the covariance matrices.
We refer the reader to \cite[Sec. VI, VII]{Forster17troOnmanifold} for the actual formulation of the preintegrated IMU factors $\phi_{\text{IMU}}$ and structureless factors $\phi_{l_s}$, as well as the underlying residual functions $\mathbf{r}_{\text{IMU}}$, $\mathbf{r}_{C_i^{l_s}}$.
For the projection factors $\phi_{l_c}$, we use a standard monocular and stereo reprojection error formulation as in \cite{Carlone17icra-vioAttention}.

\subsubsection{Regularity Constraints}
\label{sssec:regularity_constraints}

For the regularity factors $\phi_{\mathcal{R}}$, we use a co-planarity constraint between a landmark $\bm\rho_{l_c}\in\mathbb{R}^3$ and a plane $\pi = \{\bm{n}, d\}$, where $\bm{n}$ is the normal of the plane, which lives in the $\Stwo\doteq\{\mathbf{n} = (n_x, n_y, n_z)^T \big| \|\mathbf{n}\| = 1\}$ manifold, and $d\in\mathbb{R}$ the distance to the origin:
$\textstyle\residual_{\mathcal{R}} = \bm{n} \cdot \bm{\rho}_{l_c} - d$.
Representing a plane by its normal and distance to the origin is an over-parametrization that will lead to an information matrix that is singular.
This is not amenable for Gauss-Newton optimization, since it leads to singularities in the normal equations~\cite{Kaess15icra}.
%which requires the inverse of the information matrix.
To avoid the over-parametrization problem, we optimize in the tangent space $T_{\Normal}S^2 \sim \Real^{2}$ % \doteq \big\{\hat\xi \in \Real^3 | \Normal^T \hat\xi = 0\big\}$
of $S^2$ and define a suitable retraction $\calR_{\Normal}(\bm{v}): T_{\Normal}S^2 \in \mathbb{R}^2 \rightarrow \Stwo$
to map changes in the tangent space to changes to the normals in $\Stwo$~\cite{Forster17troOnmanifold}. % to obtain a minimal parametrization for the
In other words, we rewrite the residuals as:
\begin{equation}
  % \residual_\mathcal{R}(\Normal, d) = \Normal^\transpose \cdot \landmark - d \Leftrightarrow
  \residual_\mathcal{R}(\bm{v},d) = \calR_{\Normal}(\bm{v})^\transpose \cdot \landmark - d
  \label{eq:coplanarity_constraint}
\end{equation}
and optimize with respect to the minimal parametrization $\bm{v}$.
 This is similar to the proposal of Kaess~\cite{Kaess15icra}, but we work on the manifold $\Stwo$, while Kaess adopts a quaternion parametrization.
Note that, a single co-planarity constraint, as defined in \cref{eq:coplanarity_constraint}, is not sufficient to constrain a plane variable, and a minimum of three are needed instead.
Nevertheless, degenerate configurations exist, e.g.~three landmarks on a line would not fully constrain a plane.
Therefore, we ensure that a plane candidate has a minimum number of constraints before adding it to the optimization problem.
%\subsubsection{Robust Cost Functions}
%\label{sssec:robust_cost_functions}
\TODO{Do we need to mention robust cost functions at all? I think so, reviews were picky about outliers!}
%Running a feature-based VIO pipeline over planar surfaces, such as walls, has the inconvenience that few features are usually present (i.e. textureless walls), or features are easy to wrongly associate against each other (i.e. a wall with high-frequency texture).
%This can lead to certain estimated landmarks being plainly wrong (outliers), and can consequently corrupt the estimated variables in the optimization problem.
%This is especially problematic in least-squares optimization, which is particularly sensitive to outliers.

%For this reason, we will be using a robust cost function for both the projection and regularity residuals.
%In particular, we found the Huber norm to yield the best results, while we also experimented with the Tukey norm.
%More details on the norm's formulations can be found in \cite{Zhang97ivc}.
%\TODO{Someone may ask why we did not use a robust cost function for structureless factors? And I do not fully understand the underlying mechanism to avoid outliers as it is in our pipeline right now for smart factors...}

%%%%%%%%%%%%%%%%%%%%%%%%%%%%%%%%%%%%%%%%%%%%%%%%%%%%%%%%%%%%%%%%%%%%%%%%%%%%%%%%%%%%
