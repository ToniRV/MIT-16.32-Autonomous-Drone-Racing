\documentclass[letterpaper, 10 pt, journal, twoside]{IEEEtran}  % Use this command for final RAL version
%\documentclass[a4paper, 10pt, conference]{ieeeconf}      % Use this line for a4 paper

\IEEEoverridecommandlockouts                              % This command is only needed if
                                                          % you want to use the \thanks command

%\overrideIEEEmargins % Needed to meet printer requirements.

\usepackage{mathtools} 
\usepackage{graphicx,subcaption} % to use subfigure
\graphicspath{{img/}}
% To import svg, and automatic workflow with inkscape.
% But including svg also includes subfig which seems to collide with subfigure
% This gives the error main.tex|143 error| Undefined control sequence. \sf@counterlist
% Just remove deprecated subfigure, and use subcaptions instead...
%\usepackage[svgpath=img/]{svg}
% To import pgf figures.
\usepackage{pgf}
% To use svgscale before calling input on a pdf_tex file or svg
\usepackage{calc}
% To solve the issue of relative paths inside pgf.
\newcommand\inputpgf[2]{{
\let\pgfimageWithoutPath\pgfimage
\renewcommand{\pgfimage}[2][]{\pgfimageWithoutPath[##1]{#1/##2}}
\input{#1/#2}
}}

\usepackage{todonotes}
%\global\long\def\todoblue#1{\todo[inline,backgroundcolor=cyan]{#1}}
%\global\long\def\todogreen#1{\todo[inline,backgroundcolor=green]{#1}}
\usepackage{soul}

% To solve a bug with titlesec, and with makeidx
\usepackage{etoolbox}

\usepackage{amsmath} % To get math functionality
% Use amsmath to create my own definition equality symbol
\newcommand\defeq{\mathrel{\overset{\makebox[0pt]{\mbox{\tiny def}}}{=}}}
\usepackage{amssymb} % For nice math symbols
\usepackage{bm} % Better than \boldsymbol, use \bm
\usepackage[export]{adjustbox}% http://ctan.org/pkg/adjustbox
% To add tables with even width
\usepackage{tabularx}
\newcolumntype{Y}{>{\centering\arraybackslash}X} % To make columns content centered
\usepackage{tikz} % To draw
\usetikzlibrary{bayesnet} % To draw factor graphs in latex.

% Set variables for factor graph distances.
\newlength{\state} % Edge distance btw states
\setlength{\state}{1cm}

% For URLs
\usepackage{url}
% For mulicolumn
\usepackage{array}
% For copyright letters
\usepackage{textcomp}
% To avoid commands ignoring space after
\usepackage{xspace}
% To do nice minimalistic tables
% and better spacing above and below the various rules in the table
\usepackage{booktabs}
% To have footnotes in tables
\usepackage{tablefootnote}
% To allow for letters in enumerate (i, ii, ...)
\usepackage[shortlabels]{enumitem}

% For links
\usepackage{hyperref}
% Nice links and urls
\hypersetup{
    colorlinks,
    linkcolor={red!50!black},
    citecolor={blue!50!black},
    urlcolor={blue!80!black}
}

% Cleverer referencing
% Hyperref must be set before cleveref or cref won't work
\usepackage{cleveref}
%\crefformat{section}{\S#2#1#3} % Change section word for symbol
%\crefformat{subsection}{\S#2#1#3}
%\crefformat{subsubsection}{\S#2#1#3}

% Increase secnumdepth to also be able to reference paragraphs
\setcounter{secnumdepth}{6}

% For citations to be ordered
\usepackage[sort]{cite} % sorting is default actually

% useful comments
\usepackage{color}
\newcommand{\Rebuttal}[1]{\textcolor{black}{#1}}
\definecolor{orange}{RGB}{255,127,0}
\newcommand{\TR}[1]{\textcolor{orange}{{\\ \bf TR:}~#1}\\}
\newcommand{\LC}[1]{\textcolor{red}{{\bf LC:}~#1}}

% COLORS
\newcommand{\blue}[1]{{\color{blue}#1}}
\newcommand{\green}[1]{{\color{green}#1}}
\newcommand{\red}[1]{{\color{red}#1}}

% TO MANAGE REFERENCES
%============================================================================
\newcommand{\linkToPdf}[1]{\href{#1}{\blue{(pdf)}}}
\newcommand{\linkToPpt}[1]{\href{#1}{\blue{(ppt)}}}
\newcommand{\linkToCode}[1]{\href{#1}{\blue{(code)}}}
\newcommand{\linkToWeb}[1]{\href{#1}{\blue{(web)}}}
\newcommand{\linkToVideo}[1]{\href{#1}{\blue{(video)}}}
\newcommand{\award}[1]{\xspace} % {{\red{#1}}} % omit awards

\newcommand{\mysubsection}[1]{{\bf#1.}} % \subsection{}

\title{Optimal Control and Trajectory Estimation of a Nonlinear Quadrotor}
\title{Are we ready for Drone Racing? \\ Optimal Control and Trajectory Estimation}

% Paper headers
\markboth{IEEE Intelligent Robots and Systems. Preprint Version.}
{Rosinol \MakeLowercase{\textit{et al.}}: Optimal Control and Trajectory Estimation of a Nonlinear Quadrotor}
% Use only for final RAL version

\author{Antoni Rosinol$^{1}$
\thanks{$^{1}$A.\,Rosinol is with the Laboratory for Information \& Decision Systems (LIDS), Massachusetts Institute of Technology, Cambridge, MA, USA,

\sf arosinol@mit.edu}}

%\thanks{Manuscript received: September, 10, 2017; Revised December, 2, 2017; Accepted December, 4, 2017.} %Use only for final RAL version
%Use only for final RAL version.

\newcommand{\TODO}[1]{
\textcolor{orange}{{\textbf{TODO:}}~#1} % Comment this line to remove TODOs
}

\input{./chapters/shortcuts} % Shortcuts used by Forster15icra.

\begin{document}

\maketitle
%\thispagestyle{empty}
%\pagestyle{empty}

%!TEX root = ../main.tex
\begin{abstract}
    We present a suite of algorithms and simulators with open-source implementations suitable for general Robotics development.
    As robotics is starting to have an increasingly important role with applications such as drone delivery and autonomous cars, it is ever more important to test, refine and provide a proof-of-concept of our algorithmic designs in a fast yet reliable way.
    With the advent of powerful desktop computers with multiple graphic cards and high-performance processors, we have the possibility to accurately simulate worlds that are always closer to being indistinguishable from the real world.
    Although we still have not reach the level where one might discard completely a real-life demonstration for the sake of proving the validity of a system, we are getting there at an exponential pace.
    Soon we might reach the point where we will have difficulties to decide what is real from what is computer-generated. If the reader is not convinced that this will happen, we suggest checking the latest developments in Generative Adversarial Networks \TODO{cite}.

    Classical implementations of Visual-Inertial Odometry (VIO) algorithms ignore semantic information of the scene, as they rely solely on sparse landmarks.
    Nevertheless, recent work has shown the advantage of using richer representations of the scene, such as 3D meshes, to extract higher-level information such as structural regularities.
    In this work, we show that a 3D mesh of the scene can be further utilized to accommodate semantic information, which enhances the mapping side of a classical VIO beyond a sparse and uninformative point-cloud.
    Towards this end, we use recent work on semidefinite programming and conditional random fields to generate semantic information in real-time on a single-core CPU.
\end{abstract}

% Keywords appear just beneath the abstract. Use only for final version.
\begin{IEEEkeywords}
Vision-Based Navigation, Semantic Segmentation.
\end{IEEEkeywords}

\vspace{-2em}
\section*{Supplementary Material}
Videos of the experiments: \TODO{Add video url}



%!TEX root = ../main.tex

\section{Introduction}
\label{sec:introduction}

% Drop letter for first word of the Introduction
% Use only for final version.
 \IEEEPARstart{T}{he} advent of drone racing competitions such as \TODO{cite alpha pilot} requires the estimation of the fastest possible trajectory through a given race track.
 Finding this optimal trajectory can be useful for human pilots, and their tacticians, to evaluate their performance and improve upon it.
 For example, the teams in the Red Bull Air Race \TODO{cite redbull and SH} make extensive use of such information to train and perfect their manouevers.
 In this same way, drones can autonomously follow a dynamically feasible trajectory with high accuracy.
 Therefore being able to infer the optimal feasible trajectory could potentially make fully-autonomous drones beat human pilots.
 We are still not there, but great progress in this direction is being made.

{\bf Contributions.}
\begin{itemize}
  \item Formulation of the minimum-time problem for the specific problem of drone-racing through multiple gates.
  \item Presentation of a means to solve the problem using readily available software.
\end{itemize}

{\bf Paper Structure.}
Section~\ref{sec:mathematical_formulation} presents the mathematical formulation of our approach, and discusses the implementation of
Section~\ref{sec:results} reports and discusses the experimental results and comparison against related work. Section~\ref{sec:conclusions} concludes the paper.

%%!TEX root = ../main.tex

\section{Related Work}
\label{sec:related_work}

\subsection{}
\label{ssec:sota_vin}

\subsection{Optimal Control}
\label{ssec:sota_optimal_control}
%!TEX root = ../main.tex

\section{Approach}
\label{sec:mathematical_formulation}

We formulate our drone race optimization problem as a minimum time optimal control problem with nonlinear dynamics and constraints.
Therefore, we must define both the dynamics of a drone and the constraints to pass through each gate.
\Cref{ssec:model}, \ref{ssec:dynamics} and \ref{ssec:drag} present the nonlinear dynamics for a quadcopter.
\Cref{ssec:state_space_model} casts the quadcopter dynamics using a state-space model.
The state-space formulation allows us to present in a compact form the optimization problem that we detail in \cref{ssec:optimization_problem}.

\subsection{Mathematical Model of a Quadcopter}
\label{ssec:model}

The absolute position $\boldsymbol{\xi}$ of the drone is given in an Euclidean inertial frame, referred to as \textit{world frame} ($G$), by: $x, y, z$.
This corresponds to the 3D coordinates of the center of mass of the drone.
The attitude $\boldsymbol{\eta}$ of the drone with respect to the world frame is given using Euler angles: roll $\phi$, pitch $\theta$ and yaw $\psi$.

$$\boldsymbol{\xi}=\left[ \begin{array}{l}{x} \\ {y} \\ {z}\end{array}\right],
 \quad \boldsymbol{\eta}=\left[ \begin{array}{l}{\phi} \\ {\theta} \\ {\psi}\end{array}\right]$$

\Cref{fig:reference_frames} shows the frame of reference of the drone, referred to as \textit{body frame} ($B$), with respect to the world frame.

Linear velocities (body frame): $\boldsymbol{V}_{B}$

Angular velocities (body frame): $\boldsymbol{\nu}$

$$\boldsymbol{V}_{B}=\left[ \begin{array}{c}{v_{x, B}} \\ {v_{y, B}} \\ {v_{z, B}}\end{array}\right], \quad \boldsymbol{\nu}=\left[ \begin{array}{l}{p} \\ {q} \\ {r}\end{array}\right]$$

The rotation matrix from the body frame $B$ to the world frame $G$ is given by:
$\mathbf{X}^{B}=\mathbf{R}_{G}^{B} \mathbf{X}^{G}=\mathbf{R}(\phi) \mathbf{R}(\theta) \mathbf{R}(\psi) \mathbf{X}^{G}$

$$\boldsymbol{R}_B^G=\boldsymbol{R}=\left[ \begin{array}{ccc}{C_{\psi} C_{\theta}} & {C_{\psi} S_{\theta} S_{\phi}-S_{\psi} C_{\phi}} & {C_{\psi} S_{\theta} C_{\phi}+S_{\psi} S_{\phi}} \\ {S_{\psi} C_{\theta}} & {S_{\psi} S_{\theta} S_{\phi}+C_{\psi} C_{\phi}} & {S_{\psi} S_{\theta} C_{\phi}-C_{\psi} S_{\phi}} \\ {-S_{\theta}} & {C_{\theta} S_{\phi}} & {C_{\theta} C_{\phi}}\end{array}\right]$$

where $S_{x}=\sin (x)$ and $C_{x}=\cos (x)$.

Time derivative of this rotation matrix provides us with the angular velocities. These are not simply the time derivatives of the independent Euler angles, instead:

$\dot{\boldsymbol{\eta}}=\boldsymbol{W}_{\eta}^{-1} \boldsymbol{\nu}$, \hfill $\left[ \begin{array}{c}{\dot{\phi}} \\ {\dot{\theta}} \\ {\dot{\psi}}\end{array}\right]=\left[ \begin{array}{ccc}{1} & {S_{\phi} T_{\theta}} & {C_{\phi} T_{\theta}} \\ {0} & {C_{\phi}} & {-S_{\phi}} \\ {0} & {S_{\phi} / C_{\theta}} & {C_{\phi} / C_{\theta}}\end{array}\right] \left[ \begin{array}{l}{p} \\ {q} \\ {r}\end{array}\right]$

Conversely,

$\boldsymbol{\nu}=\boldsymbol{W}_{\eta} \dot{\boldsymbol{\eta}}, \hfill \left[ \begin{array}{l}{p} \\ {q} \\ {r}\end{array}\right]=\left[ \begin{array}{ccc}{1} & {0} & {-S_{\theta}} \\ {0} & {C_{\phi}} & {C_{\theta} S_{\phi}} \\ {0} & {-S_{\phi}} & {C_{\theta} C_{\phi}}\end{array}\right] \left[ \begin{array}{c}{\dot{\phi}} \\ {\dot{\theta}} \\ {\dot{\psi}}\end{array}\right]$
  
where $T_{x}=\tan (x)$.
The matrix $\boldsymbol{W}_{\eta}$ is invertible if $\theta \neq(2 k-1) \phi / 2,(k \in \mathbb{Z})$

\begin{table}[htbp]
  \begin{tabular}{|c|p{0.8\linewidth}|}
    \hline \text{Symbol} & \text{Definition} \\
    \hline 
    {$\boldsymbol{\xi}$} & {Absolute position (inertial frame)} \\
    {$\boldsymbol{\eta}$} & {Attitude (inertial frame)} \\
    {$\boldsymbol{V}_B$} & {Linear velocities (body frame)} \\
    {$\boldsymbol{\nu}$} & {Angular velocities (body frame)} \\
    {$\boldsymbol{\mathrm{R}}$} & {Rotation matrix from body to inertial frame} \\
    {$\boldsymbol{\mathrm{W}_\eta}$} & {Transformation matrix for angular velocities from inertial to body frame} \\
    {$\boldsymbol{I}$} & {Inertia matrix} \\
    {$\boldsymbol{G}$} & {Gravity} \\
    \hline
  \end{tabular}
  \caption{Definitions and Notations}
\end{table}

\textbf{Assumptions}:
\begin{itemize}
  \item The quadrotor structure is rigid and symmetrical with a center of mass aligned with the
  center of the body frame of the vehicle. The four arms of the quadcopter are aligned with the body x- and y-axes.
  Therefore, the inertia matrix $\textbf{I}$ is diagonal:

  $$\boldsymbol{I}=\left[ \begin{array}{ccc}{I_{x x}} & {0} & {0} \\ {0} & {I_{y y}} & {0} \\ {0} & {0} & {I_{z z}}\end{array}\right]$$

    We further have that due to symmetry, that the inertial components in the x- and y-axes are equal: $I_{xx}=I_{yy}$.

  \item The thrust and drag of each motor is proportional to the square of the motor velocity.
  \item The propellers are considered to be rigid and therefore blade flapping is negligible
    (deformation of propeller blades due to high velocities and flexible material). 
  \item The Earth is flat and non-rotating (difference of gravity by altitude or the spin of the earth
  is negligible).
  \item Ground effects that increase lift and decrease aerodynamic drag when flying close to the ground is considered negligible.
\end{itemize}

These assumptions and basic dynamics lead to the model used in this work.

\begin{figure}[htbp]
  \centering
  \includegraphics[width=\linewidth]{img/reference_frames.png}
  \caption{The inertial and body frames of a quadcopter using the East North Up convention (ENU). Figure from \cite{aalto}.}
  \label{fig:reference_frames}
\end{figure}

\subsection{Quadcopter Dynamics}
\label{ssec:dynamics}

The following derivations are a summary of the equations in \cite{aalto}.
\begin{itemize}
  \item The angular velocity of rotor $i,$ denoted with $\omega_{i},$ creates force $f_{i}$ in the direction of
the rotor axis: 
$$f_{i}=k \omega_{i}^{2}$$

where the lift constant is $k$.
The combined forces of rotors create thrust $T$ in the direction of the body z-axis (represented as $\boldsymbol{T}_{B}$).

$$T=\sum_{i=1}^{4} f_{i}=k \sum_{i=1}^{4} \omega_{i}^{2}, \quad \quad \boldsymbol{T}_{B}=\left[ \begin{array}{c}{0} \\ {0} \\ {T}\end{array}\right]$$

  \item The angular velocity and acceleration of the rotor also creates torque $\tau_{M_i}$ around the rotor axis: 
  $$I_{M} \dot{\omega}_{i} = \tau_{M_{i}} - b \omega_{i}^{2}$$
  in which the drag constant is $b$ and the inertia moment of the
rotor is $I_{M}$. 
Usually the acceleration of the rotor ($\dot{\omega}_{i}$) is considered small and thus it is omitted.
  This holds when we assume that the quadrotor is operating in stable flight and that the propellers are maintaining a constant thrust and not accelerating.
  This assumption results in the torque about the global z axis being equal to the torque due to drag.

Torque $\boldsymbol{\tau_B}$ consists of the torques on the three axis $\tau_\phi, \tau_\theta, \tau_\psi$:

$$
\boldsymbol{\tau}_{B}=\left[ \begin{array}{c}{\tau_{\phi}} \\ {\tau_{\theta}} \\ {\tau_{\psi}}\end{array}\right]=
\left[ \begin{array}{c}{l k\left(-\omega_{2}^{2}+\omega_{4}^{2}\right)} \\ {l k\left(-\omega_{1}^{2}+\omega_{3}^{2}\right)} \\ {\sum_{i=1}^{4} (-1)^{i+1} \tau_{M_{i}}}\end{array}\right]
  $$
  $$
  =
\left[ \begin{array}{c}
{l k\left(-\omega_{2}^{2}+\omega_{4}^{2}\right)} \\
{l k\left(-\omega_{1}^{2}+\omega_{3}^{2}\right)} \\
{b(\omega_{1}^{2}-\omega_{2}^{2}+\omega_{3}^{2}-\omega_{4}^{2})}\end{array}\right]
  $$

where $l$ is the distance between the rotor and the center of mass of the quadcopter.

\end{itemize}


\subsubsection{Translational dynamics}

Newton's second law: 
$$m \ddot{\boldsymbol{\xi}}=\boldsymbol{G}+\boldsymbol{R} \boldsymbol{T}_{B}$$

$$\left[ \begin{array}{c}{\ddot{x}} \\ {\ddot{y}} \\ {\ddot{z}}\end{array}\right]=-g \left[ \begin{array}{l}{0} \\ {0} \\ {1}\end{array}\right]+\frac{T}{m} \left[ \begin{array}{c}{C_{\psi} S_{\theta} C_{\phi}+S_{\psi} S_{\phi}} \\ {S_{\psi} S_{\theta} C_{\phi}-C_{\psi} S_{\phi}} \\ {C_{\theta} C_{\phi}}\end{array}\right]$$

\subsubsection{Rotational dynamics}
The rotational equations of motion are defined in the body frame so that the rotations can be computed about the quadrotor’s center and not the center of the global coordinate frame.
Applying Euler's second law:
$$\boldsymbol{I} \dot{\boldsymbol{\nu}}+\boldsymbol{\nu} \times(\boldsymbol{I} \boldsymbol{\nu})+\mathbf{\Gamma}=\boldsymbol{\tau}$$
where we have:
\begin{itemize}
  \item Angular acceleration of the inertia: $\boldsymbol{I\dot{v}}$
  \item Centripetal forces: $\boldsymbol{\nu}\times(\boldsymbol{I \nu})$
  \item Gyroscopic forces: $\boldsymbol{\Gamma}$
  \item External torque: $\boldsymbol{\tau}_B$
\end{itemize}

Replacing terms by their definitions, multiplying both sides by $\boldsymbol{I}^{-1}$, and rearranging:
$$\boldsymbol{\dot{\nu}}=\boldsymbol{I}^{-1}\left(-\left[ \begin{array}{c}{p} \\ {q} \\ {r}\end{array}\right] \times \left[ \begin{array}{c}{I_{x x} p} \\ {I_{y y} q} \\ {I_{z z} r}\end{array}\right]-I_{r} \left[ \begin{array}{c}{p} \\ {q} \\ {r}\end{array}\right] \times \left[ \begin{array}{c}{0} \\ {0} \\ {1}\end{array}\right] \omega_{\Gamma}+\boldsymbol{\tau}_B\right)$$

$$
\begin{array}{c}
\left[ \begin{array}{c}{\dot{p}} \\ {\dot{q}} \\ {\dot{r}}\end{array}\right]= 
\left[ \begin{array}{c}{\left(I_{y y}-I_{z z}\right) q r / I_{x x}} \\ {\left(I_{z z}-I_{x x}\right) p r / I_{y y}} \\ {(I_{x x}-I_{y y}) p q / I_{z z}}\end{array}\right] 
- I_{r} \left[ \begin{array}{c}{q / I_{x x}} \\ {-p / I_{y y}} \\ {0}\end{array}\right] \omega_{\Gamma} \\
+ \left[ \begin{array}{c}{\tau_{\phi} / I_{x x}} \\ {\tau_{\theta} / I_{y y}} \\ {\tau_{\psi} / I_{z z}}\end{array}\right]
\end{array}
  $$

  where $\omega_\Gamma = \omega_1 - \omega_2 + \omega_3 - \omega_4$, and $I_r$ is the rotor moment of inertia.


The angular accelerations $\ddot{\boldsymbol{\eta}}$ in world frame are given by the time derivatives of the angular velocities $\left(\dot{\boldsymbol{\eta}} = \boldsymbol{W}_{\eta}^{-1} \boldsymbol{\nu}\right)$,

$$\ddot{\boldsymbol{\eta}}=\frac{\mathrm{d}}{\mathrm{d} t}\left(\boldsymbol{W}_{\eta}^{-1} \boldsymbol{\nu}\right) =\frac{\mathrm{d}}{\mathrm{d} t}\left(\boldsymbol{W}_{\eta}^{-1}\right) \boldsymbol{\nu}+\boldsymbol{W}_{\eta}^{-1} \dot{\boldsymbol{\nu}} = $$

$$
\begin{array}{c}
\left[ \begin{array}{cccc}{0} & {\dot{\phi} C_{\phi} T_{\theta}+\dot{\theta} S_{\phi} / C_{\theta}^{2}} & {-\dot{\phi} S_{\phi} C_{\theta}+\dot{\theta} C_{\phi} / C_{\theta}^{2}} \\ {0} & {-\dot{\phi} S_{\phi}} & {-\dot{\phi} C_{\phi}} \\ {0} & {\dot{\phi} C_{\phi} / C_{\theta}+\dot{\phi} S_{\phi} T_{\theta} / C_{\theta}} & {-\dot{\phi} S_{\phi} / C_{\theta}+\dot{\theta} C_{\phi} T_{\theta} / C_{\theta}}\end{array}\right] \boldsymbol{\nu}
\\ + \boldsymbol{W_{\eta}^{-1}} \dot{\boldsymbol{\nu}}.
\end{array}
  $$

At this point, it is common to do a simplification by setting $[\dot{\phi} \quad \dot{\theta} \quad \dot{\psi}]^{T}=\left[ \begin{array}{lll}{p} & {q} & {r}\end{array}\right]^{T}$, which holds true for small angles of movement \cite{Sabatino2015}.
  Nevertheless, in this work we omit such a simplification.

\subsection{Aerodynamic Drag}
\label{ssec:drag}

  We include the drag force generated by the air resistance. For this, we add a diagonal coefficient matrix that associates the linear velocities to the force slowing down the movement

  $$
    \begin{array}{l}
      \left[ \begin{array}{l}{\ddot{x}} \\ {\ddot{y}} \\ {\ddot{z}}\end{array}\right]=-g \left[ \begin{array}{l}{0} \\ {0} \\ {1}\end{array}\right]+\frac{T}{m} \left[ \begin{array}{c}{C_{\psi} S_{\theta} C_{\phi}+S_{\psi} S_{\phi}} \\ {S_{\psi} S_{\theta} C_{\phi}-C_{\psi} S_{\phi}} \\ {C_{\theta} C_{\phi}}\end{array}\right] \\
      \quad \quad \quad \quad  - \frac{1}{m} \left[ \begin{array}{ccc}{A_{x}} & {0} & {0} \\ {0} & {A_{y}} & {0} \\ {0} & {0} & {A_{z}}\end{array}\right] \left[ \begin{array}{c}{\dot{x}} \\ {\dot{y}} \\ {\dot{z}}\end{array}\right]
    \end{array}
  $$

  in which $A_{x}, A_{y}$ and $A_{z}$ are the drag force coefficients for velocities in the corresponding directions of the inertial frame.
  Several other aerodynamical effects could be included in the model. For example,
dependence of thrust on angle of attack, blade flapping and airflow disruptions.
We refrain from adding these for simplicity of the model.
  We also ignore rotational drag forces since we assume rotational velocities to be small. Alternatively,
   we could add the components $\boldsymbol{\tau}_{\boldsymbol{w}}=\left[ \begin{array}{lll}{\tau_{w x}} & {\tau_{w y}} & {\tau_{w z}}\end{array}\right]^{T}$ to the overall torque.

\subsection{State-space Model}
\label{ssec:state_space_model}

We can write our nonlinear dynamics using the following state vector:

$\boldsymbol{X}^{T}=\left[\begin{array}{cccccccccccc}{x} & {y} & {z} & {\dot{x}} & {\dot{y}} & {\dot{z}} & {\phi} & {\theta} & {\psi} & {p} & {q} & {r}\end{array}\right]^{T}$

Moreover, we define our inputs as follow:
\begin{itemize}
  \item $U_1$: the resulting thrust of the four rotors.
  \item $U_2$: the difference of thrust between the motors on the $x$ axis which results in roll angle changes and subsequent movement in the lateral $x$ direction.
  \item $U_3$: the difference of thrust between the motors on the $y$ axis which results in pitch
  angle changes and subsequent movement in the lateral $y$ direction.
  \item $U_4$: the difference of torque between the clockwise and counterclockwise rotors which
  results in a moment that rotates the quadrotor around the vertical $z$ axis.
\end{itemize}

This results in the control vector $U$ defined as:
$$U=\left[ \begin{array}{l}{U_{1}} \\ {U_{2}} \\ {U_{3}} \\ {U_{4}}\end{array}\right]= \left[ \begin{array}{c}{T} \\ {\tau_{\phi}} \\ {\tau_{\theta}} \\ {\tau_{\psi}}\end{array}\right]=
\left[ \begin{array}{c}
{k \sum_{i=1}^{4} \omega_{i}^{2}} \\
{l k\left(-\omega_{2}^{2}+\omega_{4}^{2}\right)} \\ 
{l k\left(-\omega_{1}^{2}+\omega_{3}^{2}\right)} \\ 
{b(\omega_{1}^{2}-\omega_{2}^{2}+\omega_{3}^{2}-\omega_{4}^{2})}\end{array}\right]
.$$

Our nonlinear dynamics can be written as a nonlinear differential equation of the states $\boldsymbol{X}$ and control inputs $\boldsymbol{U}$:
\begin{equation}
  \label{eq:dynamics}
  \dot{\boldsymbol{X}}=f(\boldsymbol{X}, \boldsymbol{U}).
\end{equation}

Or, more precisely:
$$
\dot{\boldsymbol{X}} = 
\left[
\begin{array}{c}
\dot{x} \\
{}\\
\dot{y} \\
{}\\
\dot{z} \\
{}\\
{}\\
\ddot{x} \\
{}\\
\ddot{y} \\
{}\\
\ddot{z} \\
{}\\
{}\\
\dot{\phi} \\
{}\\
\dot{\theta} \\
{}\\
\dot{\psi} \\
{}\\
{}\\
\dot{p} \\
{}\\
\dot{q} \\
{}\\
\dot{r}
\end{array}
\right]
=
\left[
\begin{array}{l}

\dot{x} \\
{}\\
\dot{y} \\
{}\\
\dot{z} \\
{}\\
{}\\
  \frac{T}{m}[S_{\phi} S_{\psi}+C_{\phi} C_{\psi} S_{\theta}] - \frac{A_x}{m} \dot{x} \\
{}\\
  \frac{T}{m}[C_{\phi} S_{\psi} S_{\theta}-C_{\psi} S_{\phi}] - \frac{A_y}{m} \dot{y} \\
{}\\
  -g+\frac{T}{m}[C_{\phi} C_{\theta}] - \frac{A_z}{m} \dot{z} \\
{}\\
{}\\
  p+r[C_{\phi} T_{\theta}]+q[S_{\phi} T_{\theta}] \\
{}\\
  q[C_{\phi}]-r[S_{\phi}] \\
{}\\
  r \frac{C_{\phi}}{C_{\theta}}+q \frac{S_{\phi}}{C_{\theta}} \\
{}\\
{}\\
\frac{I_{y}-I_{z}}{I_{x}} r q - \frac{I_r}{I_{x}} q w_{\Gamma} +\frac{\tau_{\phi}}{I_{x}} \\
{}\\
\frac{I_{z}-I_{x}}{I_{y}} p r + \frac{I_r}{I_{y}} p w_{\Gamma} +\frac{\tau_{\theta}}{I_{y}} \\
{}\\
\frac{I_{x}-I_{y}}{I_{z}} p q +\frac{\tau_{\psi}}{I_{z}}
\end{array}
\right]
=f(\boldsymbol{X}, \boldsymbol{U})
$$

\section{Minimum Time Optimal Control Problem}
\label{ssec:optimization_problem}

A P-phase optimal control problem can be stated in the following general form.
 Determine the state
$\mathbf{x}^{(p)}(t) \in \mathbb{R}^{n(p)},$ control, $\mathbf{u}^{(p)}(t) \in \mathbb{R}^{n(t)},$ initial time, $t_{0}^{(p)} \in \mathbb{R},$ final time, $t_{f}^{(p)} \in \mathbb{R},$
in each phase $p \in[1, \ldots, P],$ that minimize a given cost functional, subject to dynamic constraints:
$$\dot{\mathbf{x}}^{(p)}=\mathbf{a}^{(p)}\left[\mathbf{x}^{(p)}, \mathbf{u}^{(p)}, t^{(p)}\right], \quad(p=1, \ldots, P)$$

\textbf{Cost}: our objective is to minimize the time taken for the drone to pass through all gates in a particular order.
Therefore, the cost we try to minimize is simply the final time:

$$J=\int_{t_{0}}^{t_{f}} d t = t_f$$
where $t_0 = 0$ is fixed, but $t_f$ is free.
We consider that the drone has finished the race once it has reached the last gate.

\textbf{Nonlinear Dynamics:} are formulated using the generic form: $\dot{x}(t)= f(x(t), u(t), t).$

As detailed in \cref{eq:dynamics}, we make use of a nonlinear function of the state $\boldsymbol{X}(t)$ and the input $\boldsymbol{U}(t)$.
Although the nonlinear function $f$ may depend explicitly on time $t$, the drone dynamics do not depend directly on $t$, therefore the time dependency can be dropped.

\textbf{Phases}
The racetrack is constructed in a piecewise fashion, using phases, where each phase spans from one gate to the next one.

\TODO{Bounds vs Constraints!}

\textbf{Constraints:} to simulate a realistic drone we must constraint the control inputs to a range of possible values.
\begin{itemize}
  \item Input Constraints: generic input constraints can be formulated as: 
  $$M_{i}^{-} \leq U_{i}(t) \leq M_{i}^{+},$$
  where $M_{i}^{-}$ is the lower bound of control input $U_{i}(t)$, while $M_{i}^{+}$ is its upper bound.
  In our case, we use the bounds in \cref{tab:control_bounds}.
  \begin{table}[htbp]
    \center
    \begin{tabular}{|c|c|c|c|}
      \hline 
      Input & Definition & $M_i^{-}$ & $M_i^{+}$ \\
      \hline 
      {$U_1$} & {Thrust} & 0.0 & 43.7 \\
      {$U_2$} & {Roll rate} & & \\
      {$U_3$} & {Pitch rate} & & \\
      {$U_4$} & {Yaw rate} & & \\
      \hline
    \end{tabular}
    \caption{Constraints on control inputs.}
    \label{tab:control_bounds}
  \end{table}

  \item State Constraints: 

\end{itemize}

\textbf{Boundary Conditions:}

Given 
\begin{itemize}
  \item Initial: $n(x(t_{0}), t_{0}) = 0$
  \item Final: $m(x(t_{f}), t_{f}) = 0$
\end{itemize}

\textbf{Initial Conditions}

%!TEX root = ../main.tex

\section{Experimental Results}
\label{sec:results}

%The output of our pipeline is two fold: the pose of the camera at each timestep and a mesh-based representation of the scene.
%To quantify the performance of the pose estimate we will be using both absolute and relative error metrics.
%These metrics will give us insights, respectively, on the global and local consistency of the trajectory estimate (\cref{ssec:state_estimation}).
%For the quality of the mesh we will be using a point cloud to point cloud distance as a metric to quantify how well the mesh represents the actual scene (\cref{ssec:mapping_quality}).
%We will also assess the real-time performance of the pipeline in \cref{ssec:timing}.
We benchmark the proposed approach against the state of the art on real datasets, and evaluate
 trajectory and map estimation accuracy, as well as runtime.
We use the \Euroc dataset~\cite{Burri15ijrr}, which contains visual and inertial data recorded from an micro aerial vehicle flying indoors.
The \Euroc dataset includes eleven datasets in total, recorded in two different scenarios.
The \textit{Machine Hall} scenario (\texttt{MH}) is the interior of an industrial facility.
It contains very little (planar) regularities.
The \textit{Vicon Room} (\texttt{V}) is similar to an office room where walls, floor and ceiling are close together, and other planar surfaces are visible
(boxes, stacked mattresses).
Datasets \texttt{V1} and \texttt{V2} differ only by the position of the objects in the scene.
%Moreover, each dataset is labelled by the level of difficulty it represents for Visual-Inertial SLAM algorithms: using the adjectives ``easy", ``medium", and ``difficult".
%The difficulty is increased by simply increasing the speed of the MAV, which results in motion-blur and drastic illumination changes on the images.
%For example, dataset \texttt{MH\_03\_medium} corresponds to a dataset in the Machine Hall where the MAV was flying at moderate speeds.
Each dataset provides the ground truth trajectory of the drone, allowing us to evaluate the accuracy of our estimation.
%compare our estimated trajectory with the real one.
For the \texttt{V} datasets, we are also provided with a ground truth point cloud of the scene, which we use to evaluate the accuracy of our mesh.

\mysubsection{Compared techniques}
To assess the advantages of our proposed approach, we compare three formulations that build one on top of the other.
First, we denote as \textbf{S}, the approach that would neither use regularity factors, nor projection factors, but only use Structureless factors ($\phi_{l_s}$, in \cref{eq:factor_form_c}).
Second, we denote as \textbf{SP}, the approach which would use Structureless factors, combined with Projection factors for those landmarks that have co-planarity constraints ($\phi_{l_c}$, in \cref{eq:factor_form_c}), but without using regularity factors.
Finally, we denote as \textbf{SPR}, our proposed formulation using Structureless, Projection and Regularity factors ($\phi_{\mathcal{R}}$, in \cref{eq:factor_form_a}).
The IMU factors ($\phi_{\text{IMU}}$, in \cref{eq:factor_form_b}) are implicitly used in all three formulations.
We also compare our results with other state-of-the-art implementations in \cref{tab:ape_accuracy_comparison_sopa}.
In particular, we compare the Root Mean Squared Error (RMSE) of our pipeline against OKVIS~\cite{Leutenegger13rss}, MSCKF~\cite{Mourikis07icra},
 ROVIO~\cite{Blosch15iros}, VINS-MONO~\cite{Qin17arxiv}, and SVO-GTSAM~\cite{Forster17troOnmanifold}, using the reported values in~\cite{Delmerico18benchmark}.
We refer the reader to \cite{Delmerico18benchmark} for details on the particular implementations and set of parameters used for each algorithm.
Note that these algorithms use a monocular camera, while we use a stereo camera.
Therefore, while \cite{Delmerico18benchmark} aligns the trajectories using $\mathrm{Sim}(3)$, we use instead $\mathrm{SE}(3)$.
Nevertheless, the scale is observable for all algorithms since they use an IMU.
We only report the values for VINS-MONO when its loop-closure module is disabled.

\subsection{Localization Performance}
\label{ssec:state_estimation}

%  One of the most important outputs of a VIO algorithm is an estimate of the
The state of our optimization problem comprises the poses of the IMU, the velocities, the IMU biases, the planes' parameters, and the landmarks' positions.
In this section we start by benchmarking the quality of the trajectory estimates, which are of paramount importance for control and AR/VR applications.
 The plane and landmark estimates will be assessed in the next \cref{ssec:mapping_quality}, where we evaluate the quality of the mesh.
We will assess the quality of the plane and landmark estimates in Section~\ref{ssec:mapping_quality}.

\mysubsection{Performance Metrics: Absolute Translation Error (ATE)}
\label{ssec:absolute_pose_error}
ATE looks at the translational part of the relative pose between the ground truth pose and the corresponding estimated pose at a given timestamp.
We first align our estimated trajectory with the ground truth trajectory both temporally and spatially (in SE(3)), as explained in \cite[Sec. 4.2.1]{RosinolMT}.
We refrain from using the rotational part since the trajectory alignment ignores the orientation of the pose estimates.
\Cref{tab:ape_all_datasets_pipelines} shows the ATE for our pipeline when using the pipelines S, SP, and our proposed approach SPR on the \Euroc dataset.

First, if we look at the performance of the different algorithmic variants for the datasets \texttt{MH\_03}, \texttt{MH\_04} and \texttt{MH\_05} in \cref{tab:ape_all_datasets_pipelines}, we observe that all methods perform equally.
This is because in these datasets no structural regularities were detected.
Hence, the proposed pipeline SPR gracefully downgrades to a standard structureless VIO pipeline (S).
Second, looking at the results
 for dataset \texttt{V2\_03}, we observe that both the SP and the SPR pipeline achieve the exact same performance.
In this case, structural regularities are detected, resulting in Projection factors being used.
Nevertheless, since the number of regularities detected is not sufficient to spawn a new plane estimate,
no structural regularities are actually enforced in the factor graph.
Finally, \cref{tab:ape_all_datasets_pipelines} shows that the SPR pipeline consistently achieves better results over the rest of datasets where structural regularities are detected and enforced.
In particular, the performance of SPR increases up to 28\% on the median ATE compared to the SP pipeline for datasets with multiple planes (e.g. \texttt{V1} and \texttt{V2}).

%We also divide the MAE by the length of the trajectory in order to be able to compare the performance between different datasets:
%\begin{equation}
  %\label{eq:ape_mae_percent}
  %\mathrm{Drift (\%)} = \frac{\frac{1}{N} \sum_{i=1}^N APE_i}{L},
%\end{equation}
%where $L$ is the length of the trajectory.

%\TODO{I still find this Drift misleading at best, at worst not representative of actual drift...
%We should be integrating the distance from start to point $i$ and dividing each $APE_i$ by this amount?}


\Cref{tab:ape_accuracy_comparison_sopa} shows that our approach, using structural regularities (SPR), achieves the best results when compared with the state-of-the-art,
 on datasets with structural regularities, such as in datasets \texttt{V1\_01\_easy} and \texttt{V1\_02\_medium},
  where multiple planes are present (walls, floor).
    We observe a $19\%$ improvement compared to the next best performing algorithm (SVO-GTSAM) in dataset \texttt{V1\_01\_easy},
     and a $26\%$ improvement in dataset \texttt{V1\_02\_medium} compared to ROVIO and VINS-MONO, which achieve the next best results.
    We also see that the performance of our pipeline is on-par with other state-of-the-art approaches when no structural regularities are present, such as in datasets \texttt{MH\_04\_difficult} and \texttt{MH\_05\_difficult}.

%\TODO{Important detail that I avoided altogether is that we are using the stereo camera also, while the above pipelines are all monocular. I just didn't mention stereo because then I have to add the stereo factors which make the whole story longer (Projection factors would now be mono/stereo etc).}
%Finally, we found to be instructive to color-code the estimated trajectory with the actual ATE errors at each pose estimate; which provides insights on how quickly the state estimation degrades.
%We provide these plots in the appendix of the thesis \TODO{cite Master's Thesis}.

% APE boxplots.
%\begin{figure*}[htbp]
  %\centering     %%% not \center
  %\includegraphics[trim={0 0 0 0cm},clip,width=0.7\textwidth]{datasets_ape_boxplots.eps}
  %\caption{Comparison of the Absolute Pose Error (APE) on the \Euroc{} datasets while using Structureless factors (S), Structureless and Projection factors (SP), and our proposed approach using Structureless, Projection and Regularity factors (SPR).}
  %\label{fig:ape_all_datasets_pipelines}
%\end{figure*}

% APE table results.
\begin{table}[tb]
  \centering
  \caption{ATE for pipelines S, SP, and SPR. Our proposed approach SPR achieves the best results for all datasets where structural regularities are detected and enforced.}
  \label{tab:ape_all_datasets_pipelines}
  \begin{tabularx}{\columnwidth}{l *6{Y}}
    \toprule
    & \multicolumn{6}{c}{ATE [cm]} \\
    \cmidrule{2-7}
    & \multicolumn{2}{c}{S} & \multicolumn{2}{c}{SP} & \multicolumn{2}{c}{SPR (\textbf{Proposed})} \\
    \cmidrule(r){2-3} \cmidrule(){4-5} \cmidrule(l){6-7}
    EuRoC Sequence & Median & RMSE & Median & RMSE & Median & RMSE \\
    \midrule
                 %MH\_01\_easy & 13.7 & 15.0 & 12.4 & 15.0 & \textbf{{10.7}} & \textbf{{14.5}} \\
             %MH\_02\_easy & 12.9 & 13.1 & 17.6 & 16.7 & \textbf{{12.6}} & \textbf{{13.0}} \\
           %MH\_03\_medium & 21.0 & 21.2 & 21.0 & 21.2 & 21.0 & 21.2 \\
        %MH\_04\_difficult & 17.3 & 21.7 & 17.3 & 21.7 & 17.3 & 21.7 \\
        %MH\_05\_difficult & 21.6 & 22.6 & 21.6 & 22.6 & 21.6 & 22.6 \\
             %V1\_01\_easy & 5.6 & 6.4 & 6.2 & 7.7 & \textbf{{5.3}} & \textbf{{5.7}} \\
           %V1\_02\_medium & 7.7 & 8.9 & 8.7 & 9.4 & \textbf{{6.3}} & \textbf{{7.4}} \\
        %V1\_03\_difficult & 17.7 & 23.1 & 13.6 & 17.6 & \textbf{{13.5}} & \textbf{{16.7}} \\
             %V2\_01\_easy & 8.0 & 8.9 & 6.6 & 8.2 & \textbf{{6.3}} & \textbf{{8.1}} \\
           %V2\_02\_medium & 8.8 & 12.7 & 9.1 & 13.5 & \textbf{{7.1}} & \textbf{{10.3}} \\
        %V2\_03\_difficult & 37.9 & 41.5 & 26.0 & 27.2 & 26.0 & 27.2 \\
                 %MH\_01\_easy & 13.7 & 15.0 & 12.4 & 15.0 & \textbf{{10.7}} & \textbf{{14.5}} \\

             MH\_02\_easy & 12.9 & 13.1 & 17.6 & 16.7 & \textbf{{12.6}} & \textbf{{13.0}} \\
             MH\_03\_medium & \textbf{21.0} & \textbf{21.2} & \textbf{21.0} & \textbf{21.2} &\textbf{21.0} & \textbf{21.2} \\
             MH\_04\_difficult & \textbf{17.3} & \textbf{21.7} & \textbf{17.3} & \textbf{21.7} & \textbf{17.3} &  \textbf{21.7} \\
             MH\_05\_difficult & \textbf{21.6} & \textbf{22.6} & \textbf{21.6} & \textbf{22.6} & \textbf{21.6} &  \textbf{22.6} \\
             V1\_01\_easy & 5.6 & 6.4 & 6.2 & 7.7 & \textbf{{5.3}} & \textbf{{5.7}} \\
           V1\_02\_medium & 7.7 & 8.9 & 8.7 & 9.4 & \textbf{{6.3}} & \textbf{{7.4}} \\
        V1\_03\_difficult & 17.7 & 23.1 & 13.6 & 17.6 & \textbf{{13.5}} & \textbf{{16.7}} \\
             V2\_01\_easy & 8.0 & 8.9 & 6.6 & 8.2 & \textbf{{6.3}} & \textbf{{8.1}} \\
           V2\_02\_medium & 8.8 & 12.7 & 9.1 & 13.5 & \textbf{{7.1}} & \textbf{{10.3}} \\
           V2\_03\_difficult & 37.9 & 41.5 & \textbf{26.0} & \textbf{27.2} & \textbf{26.0} & \textbf{27.2} \\
    \bottomrule
  \end{tabularx}%
\end{table}


% V1_01_easy
%\begin{figure}[htbp]
  %\centering

  %\begin{subfigure}[c]{0.7\columnwidth}
    %\resizebox{\columnwidth}{!}{\inputpgf{./results/V1_01_easy/SP/}{plots_APE_translation_trajectory_error.pgf}}
    %\subcaption{APE translation for S+P}
  %\end{subfigure}

  %\begin{subfigure}[c]{0.7\columnwidth}
    %\resizebox{\columnwidth}{!}{\inputpgf{./results/V1_01_easy/SPR/}{plots_APE_translation_trajectory_error.pgf}}
    %\subcaption{APE translation for S+PR}
  %\end{subfigure}

  %\caption{Dataset \texttt{V1\_01\_easy}: APE translation error plotted on the trajectory estimated by VIO using Strutureless and Projection factors (S + P), against our proposed approach using also Regularity factoris (S + P + R).}
  %\label{fig:ape_trans_traj_V1_01_easy}
%\end{figure}

%\begin{figure}[htbp]
  %\centering

  %\begin{subfigure}[c]{0.7\columnwidth}
    %\resizebox{\columnwidth}{!}{\inputpgf{./results/V1_01_easy/SP/}{plots_APE_translation.pgf}}
    %\subcaption{S+P}
  %\end{subfigure}

  %\begin{subfigure}[c]{0.7\columnwidth}
    %\resizebox{\columnwidth}{!}{\inputpgf{./results/V1_01_easy/SPR/}{plots_APE_translation.pgf}}
    %\subcaption{S+P+R}
  %\end{subfigure}

  %\caption{Dataset \texttt{V1\_01\_easy}: APE translation error of VIO using Strutureless and Projection factors (S + P), against our proposed approach using Structureless, Projection and Regularity factors (S +P + R).}
  %\label{fig:ape_trans_V1_01_easy}
%\end{figure}

% APE against state-of-the-art pipelines.

\begin{table}[tb]
  \centering
  \caption{ATE's RMSE of the state-of-the-art techniques (reported values from \cite{Delmerico18benchmark}) compared to our proposed SPR pipeline, on the \Euroc dataset. A cross ($\times$) states that the pipeline failed.
  In \textbf{bold} the best result, in \textcolor{blue}{blue} the second best.}
  \label{tab:ape_accuracy_comparison_sopa}
  \begin{tabularx}{\columnwidth}{l *6{Y}}
    \toprule
    & \multicolumn{6}{c}{RMSE ATE [cm]} \\
    \cmidrule(l){2-7}
    Sequence  & OKVIS & MSCKF & ROVIO & VINS-MONO & SVO-GTSAM & \textbf{SPR} \\
    \midrule
    MH\_01\_e & 16 & 42 & 21 & 27 & \textbf{5} & \textbf{\textcolor{blue}{14}} \\
    MH\_02\_e & 22 & 45 & 25 & \textbf{\textcolor{blue}{12}} & \textbf{3} & 13 \\
    MH\_03\_m & 24 & 23 & 25 & \textbf{\textcolor{blue}{13}} & \textbf{12} & 21 \\
    MH\_04\_d & 34 & 37 & 49 & 23 & \textbf{13} & \textbf{\textcolor{blue}{22}} \\
    MH\_05\_d & 47 & 48 & 52 & 35 & \textbf{16} & \textbf{\textcolor{blue}{23}} \\
    V1\_01\_e & 9 & 34 & 10 & \textbf{\textcolor{blue}{7}} & \textbf{\textcolor{blue}{7}} & \textbf{6} \\
    V1\_02\_m & 20 & 20 & \textbf{\textcolor{blue}{10}} & \textbf{\textcolor{blue}{10}} & 11 & \textbf{7} \\
    V1\_03\_d & 24 & 67 & \textbf{\textcolor{blue}{14}} & \textbf{13} & $\times$ & 17 \\
    V2\_01\_e & 13 & 10 & 12 & \textbf{\textcolor{blue}{8}} & \textbf{7} & \textbf{\textcolor{blue}{8}} \\
    V2\_02\_m & 16 & 16 & 14 & \textbf{8} & $\times$ & \textbf{\textcolor{blue}{10}} \\
    V2\_03\_d & 29 & 113 & \textbf{14} & \textbf{\textcolor{blue}{21}} & $\times$ & 27 \\
    \bottomrule
  \end{tabularx}%
\end{table}


    %MH\_01\_e & 16 & 42 & 21 & 27 & \textbf{5} & 14.5 \\
    %MH\_02\_e & 22 & 45 & 25 & 12 & \textbf{3} & 13.0 \\
    %MH\_03\_m & 24 & 23 & 25 & 13 & \textbf{12} & 21.2 \\
    %MH\_04\_d & 34 & 37 & 49 & 23 & \textbf{13} & 21.7 \\
    %MH\_05\_d & 47 & 48 & 52 & 35 & \textbf{16} & 22.6 \\
    %V1\_01\_e & 9 & 34 & 10 & 7 & 7 & \textbf{5.7} \\
    %V1\_02\_m & 20 & 20 & 10 & 10 & 11 & \textbf{7.4} \\
    %V1\_03\_d & 24 & 67 & 14 & \textbf{13} & $\times$ & 16.7 \\
    %V2\_01\_e & 13 & 10 & 12 & 8 & \textbf{7} & 8.1 \\
    %V2\_02\_m & 16 & 16 & 14 & \textbf{8} & $\times$ & 10.3 \\
    %V2\_03\_d & 29 & 113 & \textbf{14} & 21 & $\times$ & 27.2 \\

%\TODO{Say that in datasets where there are no structural regularities we are actually comparing our non-regular pipeline implementation against the others... but ours has not been optimized...}

\mysubsection{Performance Metrics: Relative Pose Error (RPE)}
\label{ssec:relative_pose_error}
While the ATE provides information on the global consistency of the trajectory estimate, it does not provide insights on the moment in time when the erroneous estimates happen.
Instead, the Relative Pose Error (RPE) is a metric for investigating the local consistency of a SLAM trajectory.
RPE aligns the estimated and ground truth pose for a given frame $i$, and then computes the error of the estimated pose for a frame $j>i$ at a fixed distance farther along the trajectory.
We calculate the RPE from frame $i$ to $j$ in translation and rotation (absolute angular error) \cite[Sec. 4.2.3]{RosinolMT}.
As \cite{Geiger12cvpr}, we evaluate the RPE over all possible trajectories of a given length, and for different lengths.
Nevertheless, instead of calculating the mean of all RPE for a given trajectory length, we report the maximum, the minimum, the first and third quartile, as well as the median.

\mysubsection{RPE results}
In \Cref{fig:boxplot_rpe}, we show the results for dataset \texttt{V2\_02}, where we observe that using our proposed pipeline SPR, with respect to the SP pipeline, leads to: (i) an accuracy improvement of up to 50\% in translation and 30\% in rotation (based on the maximum improvement on the median of the errors), and, (ii) an average improvement over all trajectory lenghts of 20\% in translation and 15\% in rotation (for the median errors).

%\begin{figure}[htbp]
  %\centering     %%% not \center

  %\begin{subfigure}[c]{0.7\columnwidth}
    %\includegraphics[trim={0 0 0 0cm},clip,width=\columnwidth]{./results/MH_03_medium/traj_relative_errors_boxplots.eps}
    %\subcaption{\texttt{MH\_03\_medium}}
    %\label{fig:boxplot_rpe_mh_03_v2_03_subfig_a}
  %\end{subfigure}
  %\begin{subfigure}[c]{0.7\columnwidth}
    %\includegraphics[trim={0 0 0 0cm},clip,width=\columnwidth]{./results/v2_03_difficult/traj_relative_errors_boxplots.eps}
    %\subcaption{\texttt{V2\_03\_difficult}}
    %\label{fig:boxplot_rpe_mh_03_v2_03_subfig_b}
  %\end{subfigure}

  %\caption{Detailed comparison of the state estimation accuracy while using Structureless factors (S), Structureless and Projection factors (SP), and our proposed approach using Structureless, Projection and Regularity factors (SPR) on \Euroc{}'s V1\_02\_medium and V2\_02\_medium datasets.}

  %\label{fig:boxplot_rpe_mh_03_v2_03}
%\end{figure}

\begin{figure}[htbp]
  \centering     %%% not \center

    %\includegraphics[trim={0 13.5cm 0 0cm},clip,width=0.5\columnwidth]{./results/V1_02_medium/traj_relative_errors_boxplots.eps}
    %\includegraphics[trim={0.1cm 0 0.05cm 1.25cm},clip,width=0.52\columnwidth]{./results/V1_02_medium/traj_relative_errors_boxplots.eps}
    %\hspace{-1em}
    %\includegraphics[trim={1.0cm 0 0.2cm 1.25cm},clip,width=0.4605\columnwidth]{./results/V2_02_medium/traj_relative_errors_boxplots.eps}
    %\includegraphics[trim={0cm 0 0cm 0cm},clip,width=\columnwidth]{./results/V1_02_medium/traj_relative_errors_boxplots.eps}
    \includegraphics[trim={0.1cm 0 0cm 0.2cm},clip,width=0.8\columnwidth]{./results/V2_02_medium/traj_relative_errors_boxplots.eps}

  \caption{Detailed comparison of the state estimation accuracy while using pipeline S, SP, and our proposed approach SPR on different \Euroc{} datasets.}

  \label{fig:boxplot_rpe}
\end{figure}

\subsection{Mapping quality}
\label{ssec:mapping_quality}
We use the ground truth point cloud for \texttt{V1} dataset to assess the quality of the mesh by calculating its \textit{accuracy} as defined in \cite{Schoeps2017cvpr}.

\mysubsection{Performance Metric: Map Accuracy}
  \label{ssec:map_accuracy}
%   by calculating its \textit{accuracy} %(\cref{ssec:map_accuracy})
% as defined in \cite{Schoeps2017cvpr}.
Comparing a mesh with a dense point cloud can be achieved by generating a point cloud from the mesh itself, and then comparing both point clouds.
In our case, we compute a point cloud by sampling the mesh with a uniform density of $10^3$ points per square meter.
We also register the resulting point cloud to the ground truth point cloud using an iterative closest point algorithm.
%
%\TODO{Here we are not saying how we merge sparse meshes together for evaluation (we are not just appending time-horizon meshes together...)}
  With the newly registered point cloud, we can compute a cloud to cloud distance to assess the accuracy of the mesh relative to the ground truth point cloud.
  More specifically, for each point $r$ of the estimated cloud from the mesh $\mathcal{R}$, we search the nearest point in the ground truth cloud $\mathcal{G}$, and compute their Euclidean distance $d_{r \to \mathcal{G}}$:

  \begin{equation}
    \label{eq:c2c_distance}
    d_{r \to \mathcal{G}} = \min_{g \in \mathcal{G}}\left\|{r-g}\right\|_{2} \quad\text{for}\quad r \in \mathcal{R}.
  \end{equation}

  \Cref{tab:mesh_accuracy_stats_comparison} shows that both the mean and the standard deviation of the distance from the mesh to the ground truth point cloud (\cref{eq:c2c_distance}) decreases when enforcing structural regularities, as done in the SPR pipeline.
  On average, each point sampled on the mesh generated by the SPR pipeline is $0.5$ cm closer to the ground truth point cloud than the points sampled on the mesh generated by the SP pipeline.
  Therefore, enforcing structural regularities makes the estimated mesh closer to the real scene.

  We also report the accuracy $\mathcal{A}(\tau)$, defined as the fraction of estimated points which are within a distance threshold $\tau$ of the ground truth point cloud \cite{Schoeps2017cvpr, Knapitsch2017}:
  \begin{equation}
    \label{eq:mesh_accuracy}
    \mathcal{A}(\tau) = \frac{1}{|\mathcal{R}|}\sum_{r \in \mathcal{R}}\bigg[d_{r \to \mathcal{G}} < \tau\bigg]_I \times 100,
  \end{equation}
  where $[P]_I$ is the Iverson bracket, defined as:
  \begin{equation*}
    [P]_I={\begin{cases}
        1&{\text{if }}P{\text{ is true;}}\\
        0&{\text{otherwise,}}
    \end{cases}}
  \end{equation*}

  \Cref{fig:histogram_accuracy_mesh} shows the actual error distributions for $d_{r \to \mathcal{G}}$, and the mesh accuracy $\mathcal{A}(\tau)$ for different distance thresholds $\tau$, for both the SP and the SPR pipelines respectively.
  In terms of accuracy values $\mathcal{A}(\tau)$, we can see in \cref{fig:histogram_accuracy_mesh} that the SPR pipeline consistently achieves more accurate mesh estimates (between 3\%-7\% better) for distance thresholds $\tau < 10cm$.

  As a reminder, the SP pipeline still uses the mesh to detect regularities, but, contrary to the SPR pipeline, it does not enforce the structural regularities on the landmarks.

  In \cref{fig:accuracy_mesh}, we color-encode each point on the estimated point cloud with the error distances $d_{r \to \mathcal{G}}$.
  We can observe that, when we do not enforce structural regularities, significant errors are actually present on the planar surfaces, especially on the walls (\cref{fig:accuracy_mesh} top).
  Instead, when regularities are enforced, the errors on the walls and the floor are reduced (\cref{fig:accuracy_mesh} bottom).
\TODO{Remove everything except this and the results in table 3}  A closer view on the wall itself, bottom figures (c) and (d) of \cref{fig:intro}, shows that it is visually clear that adding co-planarity constraints results in smoother walls.

% Cloud comparison table
\begin{table}[]
  \caption{Statistics for the cloud to cloud absolute distance from the mesh to the ground truth point cloud $d_{r \to \mathcal{G}}$ (\cref{eq:c2c_distance}) for dataset \texttt{V1\_01\_easy}.}
  \label{tab:mesh_accuracy_stats_comparison}
  \centering
  \begin{tabularx}{\columnwidth}{l *2{Y}}%
    \toprule
    & \multicolumn{2}{c}{VIO Type} \\
    \cmidrule(lr){2-3}
    $d_{r \to \mathcal{G}}$ Statistics & SP & SPR (\textbf{Proposed}) \\
    \cmidrule(lr){2-2} \cmidrule(lr){3-3}
    Mean $\bar{d}_{\mathcal{R}}$ [cm] & 4.9 & \textbf{4.4} \\
    Standard Deviation $\sigma_{\mathcal{R}}$ [cm] & 5.0 & \textbf{4.6} \\
    \bottomrule
  \end{tabularx}
\end{table}

% ground truth point cloud.
%\begin{figure}[tb]
%  \centering     %%% not \center
%  \includegraphics[trim={0 0 0 0cm},clip,height=0.5\columnwidth, width=\columnwidth]{./results/gt_point_cloud/V1_01_easy/frames_animation/frame_000000.png}
%  \caption{ground truth point cloud. Color-encoded for better visualization.}
%  \label{fig:ground_truth_point_cloud}
%\end{figure}

% Mesh accuracy histograms
\begin{figure}[tb]
  \centering     %%% not \center
  %\adjustbox{width=\columnwidth,trim=0.0cm 0pt 0.2ex 0pt,clip}{\resizebox{0.49\textwidth}{!}{\inputpgf{./results/S_P_Mesh/Histogram/}{Histogram_for_paper_accuracy_S_P.pgf}}}
  \adjustbox{width=\columnwidth,trim=0.1cm 0 0.1cm 0,clip}{\resizebox{\columnwidth}{!}{\inputpgf{./results/S_P_R_Mesh/Histogram/}{Histogram_for_paper_accuracy_S_P_R.pgf}}}

  %\caption{(Top) Histogram of points sampled on the mesh depending on their distance to the ground truth point cloud ($d_{r \to \mathcal{G}}$) for dataset \texttt{V1\_01\_easy} and pipelines SP (left) and SPR (right).
  %Detailed is the mesh accuracy $\mathcal{A}(\tau)$, as defined in \cref{eq:mesh_accuracy}, for different distance thresholds $\tau$. (Bottom) Color-encoded point cloud sampled from the estimated 3D mesh. The colormap is the same for all figures.}
  \caption{Histogram of points sampled on the mesh depending on their distance to the ground truth point cloud ($d_{r \to \mathcal{G}}$) for dataset \texttt{V1\_01\_easy} and pipelines SP (left) and SPR (right).
  Detailed is the mesh accuracy $\mathcal{A}(\tau)$, as defined in \cref{eq:mesh_accuracy}, for different distance thresholds $\tau$.}
  \label{fig:histogram_accuracy_mesh}
\end{figure}

\subsection{Timing}
\label{ssec:timing}

The pipelines S, SP, and SPR differ in that they try to solve an increasingly complicated problem.
While the S pipeline does not include neither the 3D landmarks nor the planes as variables in the optimization problem, the SP pipeline includes 3D landmarks, and the pipeline using regularities (SPR) further includes planes as variables.
Moreover, the SP has significantly less constraints between the variables than the SPR pipeline.
Hence, we can expect that the optimization times for the different pipelines will be each bounded by the other as $t_{S}^{opt} < t_{SP}^{opt} < t_{SPR}^{opt}$, where $t_{X}^{opt}$ is the time taken to solve the optimization problem of pipeline X.

\Cref{fig:optimization_time} shows the time taken to solve the optimization problem for each type of pipeline.
Experimentally, we observe that the optimization time follows roughly the expected distribution $t_{S}^{opt} < t_{SP}^{opt} < t_{SPR}^{opt}$.
We also observe that if the number of plane variables is large ($\sim 10^1$), and consequently the number of constraints between landmarks and planes also gets large ($\sim 10^2$), the optimization problem cannot be solved in real-time.
For example, for the keyframe index 250 in \Cref{fig:optimization_time}, we can see that a spike is present caused by the detection of multiple planes and landmarks with regularities.

\begin{figure}[tb]
  \centering     %%% not \center
    \resizebox{\columnwidth}{!}{\inputpgf{./img/}{all_timing_for_paper.pgf}}
  \caption{Comparison of the time to solve the optimization problem for pipeline S, SP, and SPR for dataset \texttt{V1\_01\_easy}.}
  \label{fig:optimization_time}
\end{figure}




%!TEX root = ../main.tex

\section{Conclusion}
\label{sec:conclusions}
\section{Supplementary Material}
Euler angle rotation matrices:

$\mathbf{R}(\psi)=\left[ \begin{array}{ccc}{\cos (\psi)} & {\sin (\psi)} & {0} \\ {-\sin (\psi)} & {\cos (\psi)} & {0} \\ {0} & {0} & {1}\end{array}\right]$

$\mathbf{R}(\theta)=\left[ \begin{array}{ccc}{\cos (\theta)} & {0} & {-\sin (\theta)} \\ {0} & {1} & {0} \\ {\sin (\theta)} & {0} & {\cos (\theta)}\end{array}\right]$

$\mathbf{R}(\phi)=\left[ \begin{array}{ccc}{1} & {0} & {0} \\ {0} & {\cos (\phi)} & {\sin (\phi)} \\ {0} & {-\sin (\phi)} & {\cos (\phi)}\end{array}\right]$

%\addtolength{\textheight}{-11cm}   % This command serves to balance the column lengths
                                  % on the last page of the document manually. It shortens
                                  % the textheight of the last page by a suitable amount.
                                  % This command does not take effect until the next page
                                  % so it should come on the page before the last. Make
                                  % sure that you do not shorten the textheight too much.


%\section*{Acknowledgments}

\bibliographystyle{IEEEtran} % use IEEEtran.bst style
\bibliography{project}

%\addtolength{\textheight}{0cm}   % This command serves to balance the column lengths
%\newpage
%\input{./chapters/appendix_content.tex}

\end{document}
